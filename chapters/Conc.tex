% ----------------------------------------------------------
\chapter{Conclusão}
% ----------------------------------------------------------

O ponto de partida para o estudo computacional dos fenômenos de impacto é a clássica mecânica do contínuo. A primeira parada dentro deste campo que foi apenas superficialmente visitado é a cinemática, portanto a descrição do movimento e das deformações. Mesmo sem uma apresentação formal do cálculo e da álgebra tensorial estes foram usados para andar por entidades tensoriais que descrevem os meios contínuos. A introdução de um conceito que não é apresentado durante os cursos de graduação em engenharia, que é a existência dos tensores de deformação finitesimal, é destacado como a parte mais importante da revisão da cinemática. Além da apresentação a importância do tensor \gls{E} foi demonstrada usando gráficos do comportamento deste em relação à sua versão linearizada \gls{eps}. \\

Depois da cinemática foram apresentados os princípios termomecânicos básicos. Eles conduzem todos os processos termomecânicos, de acordo com a mecânica do contínuo. A importância de conhecer estes princípios é semelhante à de conhecer as regras de um jogo que está sendo jogado, já que eles definem o que é ou não é permitido fisicamente. Estes princípios são todos válidos inclusive quando ondas de choque estão presentes no meio, embora sua formulação possa sofrer mudança. O conceito de ondas de choque foi introduzido no último assunto do trabalho, porém veja como a presença ou não delas afeta inclusive as considerações mais básicas, que foram apresentadas logo de início. \cite{gurtin_fried_anand_2013} faz a derivação dos mesmos conceitos termodinâmicos básicos na presença de ondas de choque, caso o leitor tenha interesse. \\ 

Dentro da revisão dos princípios termodinâmicos está um dos teoremas mais importantes e impactantes na mecânica dos sólidos, que é o teorema de Cauchy. Ele não só é fundamental para a solução do balanço de momento linear, quanto possibilita o o entendimento dos efeitos de uma entidade tão misteriosa chamada força. De acordo com \cite{gurtin_fried_anand_2013} aqueles que acreditam que a definição e o entendimento da força é algo trivial deveriam ler as obras literárias do período logo após newton. O autor inclusive cita uma frase de D'Alembert quando falava sobre as forças de newton. "Elas são entidades metafísicas obscuras capazes tão somente de espalhar escuridão sobre uma ciência inerentemente clara". Há um grande espectro de técnicas capazes de resolver o balanço de momento linear em domínios de  diferentes complexidades, porém o método dos elementos finitos é o escolhido pela indústria de defesa e também por este trabalho. A solução por elementos finitos começa pela formulação da chamada forma fraca, que gera um relaxamento na necessidade de continuidade do campo de deslocamentos. Depois disso a discretização espacial é responsável por segmentar o corpo em elementos que interagem e formam um sistema algébrico a ser resolvido no espaço. A partir daí a discretização no tempo tem responsabilidade de propagar a solução deste sistema algébrico no tempo. O resultado do sistema discretizado tanto no espaço quanto no tempo é a descrição de campos como a deformação, a tensão e o dano no material. Com esta descrição é possível confirmar ou refutar hipóteses. \\

A formulação apresentada usou a lei de Hooke como modelo constitutivo para simplificar o processo de derivação das expressões. Em um código de propagação de ondas comercial são usados muitos outros modelos para representar a tensão em função das variáveis internas do material. A plasticidade computacional foi introduzida para mostrar o ambiente teórico onde são encaixados os modelos constitutivos mais usados em simulações de impacto balístico. Então estes modelos foram apresentados, destes destacam-se o modelo de Johnson-Cook para metais e os de Johnson-Holmquist para cerâmicos, pois são simples e extremamente úteis.\\

Para realizar uma simulação basta um programa de elementos finitos em conjunto com um modelo constitutivo bem definido. Isto é fornecido por todos os programas comerciais competentes. Agora para realizar uma simulação útil são necessárias a calibração dos parâmetros da simulação e a correta interpretação de seus resultados. Tanto a interpretação quanto a calibração necessitam conhecimento do fenômeno que está acontecendo. Este conhecimento é muito mais crítico quando o impacto balístico é o fenômeno. \cite{Zukas} cita o campo como uma ciência com toques artísticos, já que usando o mesmo código para simular o mesmo fenômeno são possíveis inúmeros resultados finais, devido à diversidade de possíveis ajustes que muitas vezes precisam ser ajustados de acordo com a experiência do usuário. Além disso a simulação é apenas uma ferramenta para a pesquisa, o trabalho de \cite{kaufmann_cronin_worswick_pageau_beth_2003} mostra a dificuldade de se tirar conclusões, mesmo com a experimentação do fenômeno. Sendo assim o conhecimento dos princípios básicos e do que se deve esperar é primordial para uma simulação, já que o computador sempre apresentará resultados e cabe ao analista saber se estes são ou não confiáveis. \\